\documentclass[norsk]{article}
\usepackage[utf8]{inputenc}
\usepackage[T1]{fontenc,url}
\urlstyle{sf}
\usepackage{babel,textcomp,csquotes,varioref,graphicx}
\usepackage{amsmath,pstricks,subfigure,fixltx2e,caption,epsfig,tikz,bookman,listings}
\usepackage[backend=biber,style=numeric-comp]{biblatex}
\def\SB#1{\textsubscript{#1}}
\title{Kotetrekking og kommentarer}
\author{Sølve Rene Johnsen}

\begin{document}
\section{Ny fil: Zarray.java}
Har lagt til en ny fil har en metode som returnerer en tilfeldig fylt 
z-array som er mellom -100 og 100.

\section{Endringer i DelaunayAlg1.java}
\begin{lstlisting}
new Zarray().bygg(x,y,z); //linje 310

void kotetrekking(){

		class doubleInteger{
			int x;
			int y;
		}

		final int blueLowerBound = -100;
		final int blueUpperBound = 0;

		final int greenLowerBound = 1;
		final int greenUpperBound = 50;

		final int greyLowerBound = 51;
		final int greyUpperBound = 100;

		int a,b;
		List<doubleInteger> blueSet = new LinkedList<doubleInteger>();
		List<doubleInteger> greenSet = new LinkedList<doubleInteger>();
		List<doubleInteger> greySet = new LinkedList<doubleInteger>();

		for (int i = 0; i < n ; i++){
			a = i;
			for(int j = 0; j < delEdges[i].length; j++){
				b = delEdges[i][j];
			}
		}

	}
\end{lstlisting}
Jeg er ikke helt sikker på hvordan jeg skal gjøre matematikken som skal til
for å trekke koter, jeg har noen idéer jeg vil diskutere med deg.
\section{Endringer i DT.java}

\begin{lstlisting}
if (d.n < d.DebugLimit) g.drawString(p+"("+d.x[p]+","+d.y[p]+","+d.z[p]+")",
			          xDraw(d.x[p],0),yDraw(d.y[p],SIZE/2+1));
\end{lstlisting}
her har jeg lagt til +d.z[p] fra endring jeg har gjort under DelaunayAlg1.java.


\section{Kommentar}
Jeg har gjort lite denne uken siden jeg har vært forlover i Sandnes for 
min bestevenn, beklager det.
\end{document}
